

\section{Fields of study}

\begin{itemize}
  
\item astrophysics

  (stellar physics, stellar wind);

\item atmospheric physics and near earth space

  (exploration of middle atmosphere by remote sensing (RADAR, LIDAR), atmospheric dynamics/dynamic meteorology, geomagnetopheric physics);

\item plasma physics

  (fluid dynamics, interplanetary plasmas, magnetohydrodynamics);

  \item numerical analysis; and

  \item computer and information science

    (numerical programming, high performance computing, computational fluid dynamics).
  
\end{itemize}

\subsection{Stellar Atmospheres}

I am particularly interested in the nature of the heating of the solar corona. In solar physics, a complete understanding of the energy budget of the solar atmosphere has not yet been obtained. The coronal temperature is maintained in the megakelvin range $\sim 10^6$K, which is 3 orders of magnitude above the photospheric temperature $4\times10^3$K. Two theories have evolved as to how the corona is heated:

Mechanical wave energy transfer: MHD-waves transfer mechanical energy (initiated by granular plasma motions, possibly sub-photospheric) to the solar corona where energy is converted to heat.

Nanoflares -- a multitude of seemingly spontaneous heating events in the corona, occurring on very short time scales and small (sub-resolution) spacial scales -- heats the corona. These events are likely caused by magnetic re-connection (and their source may be the same as for MHD-waves; build up of magnetic stress due to granular plasma motions).

Much of the physics on the sun takes place in a spatial and temporal range below the instrumental capacities. For examples, many small-scale changes in the magnetic field (such as re-connection) occur too fast to be detectable. However, with the operation of the new solar observation satellite \href{http://link.springer.com/article/10.1007%2Fs11207-014-0485-y}{IRIS}, we receive data with higher spatial and temporal resolution than ever before, and exciting discoveries can be expected in the upcoming years.

\subsection{Space weather}
The state of the plasma of the near earth space (density, field strengths (B, E), fluxes and currents) conditions on the sun, in the corona, in the solar wind, at 1AU and in the sub-magnetosphere, is referred to as space weather, when observed at an near-instantaneous basis. It is known that cosmic rays are influenced by events on the sun (such as CMEs). One of these cosmic-ray impacts are called Forbush decrease, which is an effect of the cosmic ray intensity and direction in advance of an CME. By continuously monitoring the intensity and direction of cosmic rays, it is possible to predict solar events such as CMEs.

\subsection{Scientific Computing}
My experience in scientific programming include C and Fortran (for low level computations), Python and numpy/scipy packages (for scientific data processing, analysis or system programming).

I have independently developed a code (FORTRAN 95/2003) for the time-dependent solution of viscous Navier-Stokes equations, with a free surface, according to the original MAC algorithm.

I have also programmed a low-level numerical library (linear algebra), in search for improved optimization techniques (to match, or exceed, the speed of standard LAPACK or gsl libraries).

I am interested in parallel numerical computing, and the exciting challenging implications in code design.
